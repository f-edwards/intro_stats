\PassOptionsToPackage{unicode=true}{hyperref} % options for packages loaded elsewhere
\PassOptionsToPackage{hyphens}{url}
\documentclass[ignorenonframetext,]{beamer}
\IfFileExists{pgfpages.sty}{\usepackage{pgfpages}}{}
\setbeamertemplate{caption}[numbered]
\setbeamertemplate{caption label separator}{: }
\setbeamercolor{caption name}{fg=normal text.fg}
\beamertemplatenavigationsymbolsempty
\usepackage{lmodern}
\usepackage{amssymb,amsmath}
\usepackage{ifxetex,ifluatex}
\usepackage{fixltx2e} % provides \textsubscript
\ifnum 0\ifxetex 1\fi\ifluatex 1\fi=0 % if pdftex
  \usepackage[T1]{fontenc}
  \usepackage[utf8]{inputenc}
\else % if luatex or xelatex
  \ifxetex
    \usepackage{mathspec}
  \else
    \usepackage{fontspec}
\fi
\defaultfontfeatures{Ligatures=TeX,Scale=MatchLowercase}







\fi

  \usetheme[]{metropolis}






% use upquote if available, for straight quotes in verbatim environments
\IfFileExists{upquote.sty}{\usepackage{upquote}}{}
% use microtype if available
\IfFileExists{microtype.sty}{%
  \usepackage{microtype}
  \UseMicrotypeSet[protrusion]{basicmath} % disable protrusion for tt fonts
}{}


\newif\ifbibliography


\hypersetup{
      pdftitle={Regression part 4: interactions},
        pdfauthor={Frank Edwards},
          pdfborder={0 0 0},
    breaklinks=true}
%\urlstyle{same}  % Use monospace font for urls




  \usepackage{color}
  \usepackage{fancyvrb}
  \newcommand{\VerbBar}{|}
  \newcommand{\VERB}{\Verb[commandchars=\\\{\}]}
  \DefineVerbatimEnvironment{Highlighting}{Verbatim}{commandchars=\\\{\}}
  % Add ',fontsize=\small' for more characters per line
  \usepackage{framed}
  \definecolor{shadecolor}{RGB}{248,248,248}
  \newenvironment{Shaded}{\begin{snugshade}}{\end{snugshade}}
  \newcommand{\AlertTok}[1]{\textcolor[rgb]{0.94,0.16,0.16}{#1}}
  \newcommand{\AnnotationTok}[1]{\textcolor[rgb]{0.56,0.35,0.01}{\textbf{\textit{#1}}}}
  \newcommand{\AttributeTok}[1]{\textcolor[rgb]{0.13,0.29,0.53}{#1}}
  \newcommand{\BaseNTok}[1]{\textcolor[rgb]{0.00,0.00,0.81}{#1}}
  \newcommand{\BuiltInTok}[1]{#1}
  \newcommand{\CharTok}[1]{\textcolor[rgb]{0.31,0.60,0.02}{#1}}
  \newcommand{\CommentTok}[1]{\textcolor[rgb]{0.56,0.35,0.01}{\textit{#1}}}
  \newcommand{\CommentVarTok}[1]{\textcolor[rgb]{0.56,0.35,0.01}{\textbf{\textit{#1}}}}
  \newcommand{\ConstantTok}[1]{\textcolor[rgb]{0.56,0.35,0.01}{#1}}
  \newcommand{\ControlFlowTok}[1]{\textcolor[rgb]{0.13,0.29,0.53}{\textbf{#1}}}
  \newcommand{\DataTypeTok}[1]{\textcolor[rgb]{0.13,0.29,0.53}{#1}}
  \newcommand{\DecValTok}[1]{\textcolor[rgb]{0.00,0.00,0.81}{#1}}
  \newcommand{\DocumentationTok}[1]{\textcolor[rgb]{0.56,0.35,0.01}{\textbf{\textit{#1}}}}
  \newcommand{\ErrorTok}[1]{\textcolor[rgb]{0.64,0.00,0.00}{\textbf{#1}}}
  \newcommand{\ExtensionTok}[1]{#1}
  \newcommand{\FloatTok}[1]{\textcolor[rgb]{0.00,0.00,0.81}{#1}}
  \newcommand{\FunctionTok}[1]{\textcolor[rgb]{0.13,0.29,0.53}{\textbf{#1}}}
  \newcommand{\ImportTok}[1]{#1}
  \newcommand{\InformationTok}[1]{\textcolor[rgb]{0.56,0.35,0.01}{\textbf{\textit{#1}}}}
  \newcommand{\KeywordTok}[1]{\textcolor[rgb]{0.13,0.29,0.53}{\textbf{#1}}}
  \newcommand{\NormalTok}[1]{#1}
  \newcommand{\OperatorTok}[1]{\textcolor[rgb]{0.81,0.36,0.00}{\textbf{#1}}}
  \newcommand{\OtherTok}[1]{\textcolor[rgb]{0.56,0.35,0.01}{#1}}
  \newcommand{\PreprocessorTok}[1]{\textcolor[rgb]{0.56,0.35,0.01}{\textit{#1}}}
  \newcommand{\RegionMarkerTok}[1]{#1}
  \newcommand{\SpecialCharTok}[1]{\textcolor[rgb]{0.81,0.36,0.00}{\textbf{#1}}}
  \newcommand{\SpecialStringTok}[1]{\textcolor[rgb]{0.31,0.60,0.02}{#1}}
  \newcommand{\StringTok}[1]{\textcolor[rgb]{0.31,0.60,0.02}{#1}}
  \newcommand{\VariableTok}[1]{\textcolor[rgb]{0.00,0.00,0.00}{#1}}
  \newcommand{\VerbatimStringTok}[1]{\textcolor[rgb]{0.31,0.60,0.02}{#1}}
  \newcommand{\WarningTok}[1]{\textcolor[rgb]{0.56,0.35,0.01}{\textbf{\textit{#1}}}}

  \usepackage{longtable,booktabs}
  \usepackage{caption}
  % These lines are needed to make table captions work with longtable:
  \makeatletter
  \def\fnum@table{\tablename~\thetable}
  \makeatother

  \usepackage{graphicx,grffile}
  \makeatletter
  \def\maxwidth{\ifdim\Gin@nat@width>\linewidth\linewidth\else\Gin@nat@width\fi}
  \def\maxheight{\ifdim\Gin@nat@height>\textheight0.8\textheight\else\Gin@nat@height\fi}
  \makeatother
  % Scale images if necessary, so that they will not overflow the page
  % margins by default, and it is still possible to overwrite the defaults
  % using explicit options in \includegraphics[width, height, ...]{}
  \setkeys{Gin}{width=\maxwidth,height=\maxheight,keepaspectratio}

% Prevent slide breaks in the middle of a paragraph:
\widowpenalties 1 10000
\raggedbottom

  \AtBeginPart{
    \let\insertpartnumber\relax
    \let\partname\relax
    \frame{\partpage}
  }
  \AtBeginSection{
    \ifbibliography
    \else
      \let\insertsectionnumber\relax
      \let\sectionname\relax
      \frame{\sectionpage}
    \fi
  }
  \AtBeginSubsection{
    \let\insertsubsectionnumber\relax
    \let\subsectionname\relax
    \frame{\subsectionpage}
  }



\setlength{\parindent}{0pt}
\setlength{\parskip}{6pt plus 2pt minus 1pt}
\setlength{\emergencystretch}{3em}  % prevent overfull lines
\providecommand{\tightlist}{%
  \setlength{\itemsep}{0pt}\setlength{\parskip}{0pt}}

  \setcounter{secnumdepth}{0}



  \title[]{Regression part 4: interactions}


  \author[
        Frank Edwards
    ]{Frank Edwards}


\date[
      
  ]{
    }


\providecommand{\pandocbounded}[1]{#1}

\begin{document}

% Hide progress bar and footline on titlepage
  \begin{frame}[plain]
  \titlepage
  \end{frame}



\begin{frame}{The use of regression}
\phantomsection\label{the-use-of-regression}
Sometimes we use regression to estimate causal relationships (e.g.~The
Mark of a Criminal Record).

Sometimes we use regression for pure prediction (e.g.~election
forecasts)

\textbf{Sometimes we use regression to help us better understand and
describe a process that depends on many variables.}
\end{frame}

\begin{frame}{Building a model to approximate the data generating
process}
\phantomsection\label{building-a-model-to-approximate-the-data-generating-process}
\begin{enumerate}
\tightlist
\item
  Develop an explicit theoretical model
\item
  Evaluate data availability and quality
\item
  Experiment with model specification
\item
  Evaluate goodness-of-fit metrics
\item
  Evaluate the \emph{predictive distribution} relative to the
  \emph{empirical distribution}
\end{enumerate}
\end{frame}

\begin{frame}{So what processes \emph{cause} income to vary across
people?}
\phantomsection\label{so-what-processes-cause-income-to-vary-across-people}
\tiny

\pandocbounded{\includegraphics[keepaspectratio]{20_descriptive_regression_files/figure-beamer/unnamed-chunk-1-1.pdf}}

\normalsize
\end{frame}

\begin{frame}[fragile]{Let's check our data}
\phantomsection\label{lets-check-our-data}
\tiny

\begin{Shaded}
\begin{Highlighting}[]
\NormalTok{dat}\OtherTok{\textless{}{-}}\FunctionTok{read\_csv}\NormalTok{(}\StringTok{"https://www.openintro.org/data/csv/acs12.csv"}\NormalTok{)}
\DocumentationTok{\#\#\# subset to in labor force}
\NormalTok{dat }\OtherTok{\textless{}{-}}\NormalTok{ dat }\SpecialCharTok{|\textgreater{}} 
  \FunctionTok{filter}\NormalTok{(employment }\SpecialCharTok{!=} \StringTok{"not in labor force"}\NormalTok{)}
\FunctionTok{glimpse}\NormalTok{(dat)}
\end{Highlighting}
\end{Shaded}

\begin{verbatim}
## Rows: 949
## Columns: 13
## $ income       <dbl> 1700, 45000, 8600, 33500, 4000, 19000, 3400, 0, 140000, 0~
## $ employment   <chr> "employed", "employed", "employed", "employed", "employed~
## $ hrs_work     <dbl> 40, 84, 23, 55, 8, 35, 25, NA, 40, 8, 23, 72, 40, 50, 35,~
## $ race         <chr> "other", "white", "white", "white", "white", "white", "wh~
## $ age          <dbl> 35, 27, 69, 52, 67, 36, 40, 27, 35, 31, 32, 35, 51, 50, 2~
## $ gender       <chr> "female", "male", "female", "male", "female", "female", "~
## $ citizen      <chr> "yes", "yes", "yes", "yes", "yes", "yes", "yes", "yes", "~
## $ time_to_work <dbl> 15, 40, 5, 20, 10, 15, NA, NA, 30, 20, 45, 25, 10, 40, 10~
## $ lang         <chr> "other", "english", "english", "english", "english", "eng~
## $ married      <chr> "yes", "yes", "no", "yes", "yes", "yes", "no", "no", "no"~
## $ edu          <chr> "hs or lower", "hs or lower", "hs or lower", "hs or lower~
## $ disability   <chr> "yes", "no", "no", "no", "no", "no", "yes", "no", "no", "~
## $ birth_qrtr   <chr> "jul thru sep", "oct thru dec", "jul thru sep", "apr thru~
\end{verbatim}

\normalsize
\end{frame}

\begin{frame}[fragile]{The distribution of income among those in the
labor force}
\phantomsection\label{the-distribution-of-income-among-those-in-the-labor-force}
\tiny

\begin{Shaded}
\begin{Highlighting}[]
\FunctionTok{summary}\NormalTok{(dat}\SpecialCharTok{$}\NormalTok{income)}
\end{Highlighting}
\end{Shaded}

\begin{verbatim}
##    Min. 1st Qu.  Median    Mean 3rd Qu.    Max. 
##       0    6600   25200   39808   50000  450000
\end{verbatim}

\begin{Shaded}
\begin{Highlighting}[]
\FunctionTok{ggplot}\NormalTok{(dat,}
       \FunctionTok{aes}\NormalTok{(}\AttributeTok{x =}\NormalTok{ income)) }\SpecialCharTok{+} 
  \FunctionTok{geom\_histogram}\NormalTok{() }\SpecialCharTok{+} 
  \FunctionTok{scale\_x\_sqrt}\NormalTok{()}
\end{Highlighting}
\end{Shaded}

\pandocbounded{\includegraphics[keepaspectratio]{20_descriptive_regression_files/figure-beamer/unnamed-chunk-3-1.pdf}}

\normalsize
\end{frame}

\begin{frame}[fragile]{Let's check our data}
\phantomsection\label{lets-check-our-data-1}
\tiny

\begin{Shaded}
\begin{Highlighting}[]
\NormalTok{dat }\SpecialCharTok{|\textgreater{}} \FunctionTok{group\_by}\NormalTok{(race, gender) }\SpecialCharTok{|\textgreater{}} 
  \FunctionTok{summarize}\NormalTok{(}\AttributeTok{n =} \FunctionTok{n}\NormalTok{()) }\SpecialCharTok{|\textgreater{}} 
\NormalTok{  knitr}\SpecialCharTok{::}\FunctionTok{kable}\NormalTok{()}
\end{Highlighting}
\end{Shaded}

\begin{longtable}[]{@{}llr@{}}
\toprule\noalign{}
race & gender & n \\
\midrule\noalign{}
\endhead
asian & female & 14 \\
asian & male & 28 \\
black & female & 48 \\
black & male & 48 \\
other & female & 34 \\
other & male & 35 \\
white & female & 324 \\
white & male & 418 \\
\bottomrule\noalign{}
\end{longtable}

\normalsize
\end{frame}

\begin{frame}[fragile]{Fitting a preliminary model}
\phantomsection\label{fitting-a-preliminary-model}
Our theory tells us that income is a function of age, disability,
education, race, and gender. It doesn't tell us what form those function
take though!

Let's start simple and additive

\tiny

\begin{Shaded}
\begin{Highlighting}[]
\NormalTok{m0}\OtherTok{\textless{}{-}}\FunctionTok{lm}\NormalTok{(income }\SpecialCharTok{\textasciitilde{}}\NormalTok{ edu }\SpecialCharTok{+}\NormalTok{ age }\SpecialCharTok{+} 
\NormalTok{         race }\SpecialCharTok{+}\NormalTok{ disability }\SpecialCharTok{+}\NormalTok{ gender,}
       \AttributeTok{data =}\NormalTok{ dat)}
\end{Highlighting}
\end{Shaded}

\normalsize

This model can be written as

\[y_i = \beta_0 + \beta_1 edu_i + \beta_2 age_i + \beta_3 race_i + \beta_4 disability_i + \beta_5 gender_i + \varepsilon_i\]
\end{frame}

\begin{frame}[fragile]{Evaluating our model fit with R\^{}2}
\phantomsection\label{evaluating-our-model-fit-with-r2}
\tiny

\begin{verbatim}
## 
## Call:
## lm(formula = income ~ edu + age + race + disability + gender, 
##     data = dat)
## 
## Residuals:
##     Min      1Q  Median      3Q     Max 
## -115914  -23209   -4333   12883  332880 
## 
## Coefficients:
##                Estimate Std. Error t value Pr(>|t|)    
## (Intercept)     39810.9     8960.6   4.443 9.93e-06 ***
## edugrad         45547.5     5671.5   8.031 2.88e-15 ***
## eduhs or lower -18364.0     3603.7  -5.096 4.19e-07 ***
## age               603.3      108.9   5.540 3.93e-08 ***
## raceblack      -38705.4     9023.7  -4.289 1.98e-05 ***
## raceother      -39660.9     9520.7  -4.166 3.39e-05 ***
## racewhite      -29874.8     7720.1  -3.870 0.000116 ***
## disabilityyes  -16771.6     5452.3  -3.076 0.002158 ** 
## gendermale      22421.5     3165.0   7.084 2.74e-12 ***
## ---
## Signif. codes:  0 '***' 0.001 '**' 0.01 '*' 0.05 '.' 0.1 ' ' 1
## 
## Residual standard error: 48220 on 940 degrees of freedom
## Multiple R-squared:  0.247,  Adjusted R-squared:  0.2406 
## F-statistic: 38.54 on 8 and 940 DF,  p-value: < 2.2e-16
\end{verbatim}

\normalsize
\end{frame}

\begin{frame}{Proportion of variance explained}
\phantomsection\label{proportion-of-variance-explained}
The coefficient of determination, \(R^2\), provides one measure of
\emph{goodness-of-fit}.

\[R^2 = \frac{\sum (y_i - \hat{y})^2}{\sum (y_i - \bar{y})^2} \]

\(R^2\) tells us how much of the variation in \(y\) is explained by the
regression line \(y = \beta X\) compared to the line \(y = \bar{y}\)
\end{frame}

\begin{frame}[fragile]{GoF basics}
\phantomsection\label{gof-basics}
\tiny

\begin{Shaded}
\begin{Highlighting}[]
\NormalTok{mod1}\OtherTok{\textless{}{-}}\FunctionTok{lm}\NormalTok{(income }\SpecialCharTok{\textasciitilde{}}\NormalTok{ age, }\AttributeTok{data =}\NormalTok{ dat)}
\FunctionTok{summary}\NormalTok{(mod1)}\SpecialCharTok{$}\NormalTok{r.squared}
\end{Highlighting}
\end{Shaded}

\begin{verbatim}
## [1] 0.03610956
\end{verbatim}

\begin{Shaded}
\begin{Highlighting}[]
\NormalTok{mod2}\OtherTok{\textless{}{-}}\FunctionTok{lm}\NormalTok{(income }\SpecialCharTok{\textasciitilde{}}\NormalTok{ hrs\_work, }\AttributeTok{data =}\NormalTok{ dat)}
\FunctionTok{summary}\NormalTok{(mod2)}\SpecialCharTok{$}\NormalTok{r.squared}
\end{Highlighting}
\end{Shaded}

\begin{verbatim}
## [1] 0.1174225
\end{verbatim}

\normalsize

Which model is a better fit?
\end{frame}

\begin{frame}{Have we improved our fit compared to guessing the mean
(dotted line)?}
\phantomsection\label{have-we-improved-our-fit-compared-to-guessing-the-mean-dotted-line}
\tiny

\pandocbounded{\includegraphics[keepaspectratio]{20_descriptive_regression_files/figure-beamer/unnamed-chunk-8-1.pdf}}

\normalsize
\end{frame}

\begin{frame}[fragile]{GoF as reduction in error}
\phantomsection\label{gof-as-reduction-in-error}
\tiny

\begin{Shaded}
\begin{Highlighting}[]
\DocumentationTok{\#\# How much residual error is there in model 1?}
\FunctionTok{sum}\NormalTok{(mod1}\SpecialCharTok{$}\NormalTok{residuals}\SpecialCharTok{\^{}}\DecValTok{2}\NormalTok{)}
\end{Highlighting}
\end{Shaded}

\begin{verbatim}
## [1] 2.797424e+12
\end{verbatim}

\begin{Shaded}
\begin{Highlighting}[]
\DocumentationTok{\#\# and how much in model 2?}
\FunctionTok{sum}\NormalTok{(mod2}\SpecialCharTok{$}\NormalTok{residuals}\SpecialCharTok{\^{}}\DecValTok{2}\NormalTok{)}
\end{Highlighting}
\end{Shaded}

\begin{verbatim}
## [1] 2.481356e+12
\end{verbatim}

\normalsize
\end{frame}

\begin{frame}[fragile]{So let's estimate and compare some models}
\phantomsection\label{so-lets-estimate-and-compare-some-models}
\tiny

\begin{Shaded}
\begin{Highlighting}[]
\CommentTok{\# our additive model}
\NormalTok{m0}\OtherTok{\textless{}{-}}\FunctionTok{lm}\NormalTok{(income }\SpecialCharTok{\textasciitilde{}}\NormalTok{ edu }\SpecialCharTok{+}\NormalTok{ age }\SpecialCharTok{+} 
\NormalTok{         race }\SpecialCharTok{+}\NormalTok{ disability }\SpecialCharTok{+}\NormalTok{ gender,}
       \AttributeTok{data =}\NormalTok{ dat)}
\CommentTok{\# maybe education{-}\textgreater{} income varies by gender?}
\NormalTok{m1}\OtherTok{\textless{}{-}}\FunctionTok{lm}\NormalTok{(income }\SpecialCharTok{\textasciitilde{}}\NormalTok{ edu }\SpecialCharTok{*}\NormalTok{ gender }\SpecialCharTok{+} 
\NormalTok{         age }\SpecialCharTok{+}\NormalTok{ race }\SpecialCharTok{+}\NormalTok{ disability,}
       \AttributeTok{data =}\NormalTok{ dat)}

\FunctionTok{summary}\NormalTok{(m0)}\SpecialCharTok{$}\NormalTok{r.squared}
\end{Highlighting}
\end{Shaded}

\begin{verbatim}
## [1] 0.2469889
\end{verbatim}

\begin{Shaded}
\begin{Highlighting}[]
\FunctionTok{summary}\NormalTok{(m1)}\SpecialCharTok{$}\NormalTok{r.squared}
\end{Highlighting}
\end{Shaded}

\begin{verbatim}
## [1] 0.263837
\end{verbatim}

\normalsize
\end{frame}

\begin{frame}[fragile]{So let's estimate and compare some models}
\phantomsection\label{so-lets-estimate-and-compare-some-models-1}
\tiny

\begin{Shaded}
\begin{Highlighting}[]
\CommentTok{\# maybe education{-}\textgreater{} income varies by gender and race?}
\NormalTok{m2}\OtherTok{\textless{}{-}}\FunctionTok{lm}\NormalTok{(income }\SpecialCharTok{\textasciitilde{}}\NormalTok{ edu }\SpecialCharTok{*}\NormalTok{ (gender }\SpecialCharTok{+}\NormalTok{ race) }\SpecialCharTok{+} 
\NormalTok{         age }\SpecialCharTok{+}\NormalTok{ disability,}
       \AttributeTok{data =}\NormalTok{ dat)}

\FunctionTok{summary}\NormalTok{(m1)}\SpecialCharTok{$}\NormalTok{r.squared}
\end{Highlighting}
\end{Shaded}

\begin{verbatim}
## [1] 0.263837
\end{verbatim}

\begin{Shaded}
\begin{Highlighting}[]
\FunctionTok{summary}\NormalTok{(m2)}\SpecialCharTok{$}\NormalTok{r.squared}
\end{Highlighting}
\end{Shaded}

\begin{verbatim}
## [1] 0.2769639
\end{verbatim}

\normalsize
\end{frame}

\begin{frame}[fragile]{So let's estimate and compare some models}
\phantomsection\label{so-lets-estimate-and-compare-some-models-2}
\tiny

\begin{Shaded}
\begin{Highlighting}[]
\CommentTok{\# maybe education{-}\textgreater{} income varies by race/gender pairs?}
\NormalTok{m3}\OtherTok{\textless{}{-}}\FunctionTok{lm}\NormalTok{(income }\SpecialCharTok{\textasciitilde{}}\NormalTok{ edu }\SpecialCharTok{*}\NormalTok{ (gender }\SpecialCharTok{*}\NormalTok{ race) }\SpecialCharTok{+} 
\NormalTok{         age }\SpecialCharTok{+}\NormalTok{ disability,}
       \AttributeTok{data =}\NormalTok{ dat)}

\FunctionTok{summary}\NormalTok{(m3)}\SpecialCharTok{$}\NormalTok{r.squared}
\end{Highlighting}
\end{Shaded}

\begin{verbatim}
## [1] 0.287483
\end{verbatim}

\begin{Shaded}
\begin{Highlighting}[]
\FunctionTok{summary}\NormalTok{(m2)}\SpecialCharTok{$}\NormalTok{r.squared}
\end{Highlighting}
\end{Shaded}

\begin{verbatim}
## [1] 0.2769639
\end{verbatim}

\normalsize
\end{frame}

\begin{frame}[fragile]{Let's go nuts}
\phantomsection\label{lets-go-nuts}
\tiny

\begin{Shaded}
\begin{Highlighting}[]
\CommentTok{\# maybe education{-}\textgreater{} income varies by race/gender pairs?}
\NormalTok{m4}\OtherTok{\textless{}{-}}\FunctionTok{lm}\NormalTok{(income }\SpecialCharTok{\textasciitilde{}}\NormalTok{ edu }\SpecialCharTok{*}\NormalTok{ (gender }\SpecialCharTok{*}\NormalTok{ race }\SpecialCharTok{*} 
\NormalTok{         age }\SpecialCharTok{*}\NormalTok{ disability),}
       \AttributeTok{data =}\NormalTok{ dat)}

\FunctionTok{summary}\NormalTok{(m3)}\SpecialCharTok{$}\NormalTok{r.squared}
\end{Highlighting}
\end{Shaded}

\begin{verbatim}
## [1] 0.287483
\end{verbatim}

\begin{Shaded}
\begin{Highlighting}[]
\FunctionTok{summary}\NormalTok{(m4)}\SpecialCharTok{$}\NormalTok{r.squared}
\end{Highlighting}
\end{Shaded}

\begin{verbatim}
## [1] 0.321665
\end{verbatim}

\normalsize
\end{frame}

\begin{frame}[fragile]{When are we just overfitting?}
\phantomsection\label{when-are-we-just-overfitting}
The Bayesian Information Criterion (BIC) provides a check against
overfitting. It evaluates goodness of fit with a penalty for complexity
(count of model parameters), based on the log-likelihood of the model.
The first term \(k\ln(n)\) adjusts for model complexity with \(n\) as
the number of observations and \(k\) as the number of model parameters
(\(\beta\))

\[BIC = k \ln(n)  - 2\ln(\hat{L})\]

\tiny

\begin{Shaded}
\begin{Highlighting}[]
\FunctionTok{BIC}\NormalTok{(m0, m1, m2, m3, m4)}
\end{Highlighting}
\end{Shaded}

\begin{verbatim}
##    df      BIC
## m0 10 23219.69
## m1 12 23211.92
## m2 18 23235.98
## m3 27 23283.77
## m4 72 23545.61
\end{verbatim}

\normalsize
\end{frame}

\begin{frame}{Visualizing observed versus expected Model 0}
\phantomsection\label{visualizing-observed-versus-expected-model-0}
\tiny

\pandocbounded{\includegraphics[keepaspectratio]{20_descriptive_regression_files/figure-beamer/unnamed-chunk-15-1.pdf}}

\normalsize
\end{frame}

\begin{frame}{Visualizing observed versus expected Model 1}
\phantomsection\label{visualizing-observed-versus-expected-model-1}
\tiny

\pandocbounded{\includegraphics[keepaspectratio]{20_descriptive_regression_files/figure-beamer/unnamed-chunk-16-1.pdf}}

\normalsize
\end{frame}

\begin{frame}{Visualizing observed versus expected Model 2}
\phantomsection\label{visualizing-observed-versus-expected-model-2}
\tiny

\pandocbounded{\includegraphics[keepaspectratio]{20_descriptive_regression_files/figure-beamer/unnamed-chunk-17-1.pdf}}

\normalsize
\end{frame}

\begin{frame}{Visualizing observed versus expected Model 3}
\phantomsection\label{visualizing-observed-versus-expected-model-3}
\tiny

\pandocbounded{\includegraphics[keepaspectratio]{20_descriptive_regression_files/figure-beamer/unnamed-chunk-18-1.pdf}}

\normalsize
\end{frame}

\begin{frame}{Visualizing observed versus expected Model 4}
\phantomsection\label{visualizing-observed-versus-expected-model-4}
\tiny

\pandocbounded{\includegraphics[keepaspectratio]{20_descriptive_regression_files/figure-beamer/unnamed-chunk-19-1.pdf}}

\normalsize
\end{frame}

\begin{frame}{Fitted vs observed plots can be very informative: Model 0}
\phantomsection\label{fitted-vs-observed-plots-can-be-very-informative-model-0}
\tiny

\pandocbounded{\includegraphics[keepaspectratio]{20_descriptive_regression_files/figure-beamer/unnamed-chunk-20-1.pdf}}

\normalsize
\end{frame}

\begin{frame}{Fitted vs observed plots can be very informative: Model 1}
\phantomsection\label{fitted-vs-observed-plots-can-be-very-informative-model-1}
\tiny

\pandocbounded{\includegraphics[keepaspectratio]{20_descriptive_regression_files/figure-beamer/unnamed-chunk-21-1.pdf}}

\normalsize
\end{frame}

\begin{frame}{Fitted vs observed plots can be very informative: Model 2}
\phantomsection\label{fitted-vs-observed-plots-can-be-very-informative-model-2}
\tiny

\pandocbounded{\includegraphics[keepaspectratio]{20_descriptive_regression_files/figure-beamer/unnamed-chunk-22-1.pdf}}

\normalsize
\end{frame}

\begin{frame}{Fitted vs observed plots can be very informative: Model 3}
\phantomsection\label{fitted-vs-observed-plots-can-be-very-informative-model-3}
\tiny

\pandocbounded{\includegraphics[keepaspectratio]{20_descriptive_regression_files/figure-beamer/unnamed-chunk-23-1.pdf}}

\normalsize
\end{frame}

\begin{frame}{Fitted vs observed plots can be very informative: Model 4}
\phantomsection\label{fitted-vs-observed-plots-can-be-very-informative-model-4}
\tiny

\pandocbounded{\includegraphics[keepaspectratio]{20_descriptive_regression_files/figure-beamer/unnamed-chunk-24-1.pdf}}

\normalsize
\end{frame}

\begin{frame}{Which model is best?}
\phantomsection\label{which-model-is-best}
It depends on our target!

\tiny

\begin{longtable}[]{@{}lrrrr@{}}
\toprule\noalign{}
& model & r2 & BIC.df & BIC.BIC \\
\midrule\noalign{}
\endhead
m0 & 0 & 0.2469889 & 10 & 23219.69 \\
m1 & 1 & 0.2638370 & 12 & 23211.92 \\
m2 & 2 & 0.2769639 & 18 & 23235.98 \\
m3 & 3 & 0.2874830 & 27 & 23283.77 \\
m4 & 4 & 0.3216650 & 72 & 23545.61 \\
\bottomrule\noalign{}
\end{longtable}

\normalsize
\end{frame}

\begin{frame}{General advice}
\phantomsection\label{general-advice}
When fitting a model for \emph{descriptive} or \emph{predictive}
purposes

\begin{enumerate}
\tightlist
\item
  Choose predictors based on theory
\item
  Experiment with varying function forms (additive, interactive,
  nonlinear)
\item
  Compare goodness of fit using \(R^2\), but also use BIC and other
  criteria robust to overfitting (leave-one-out is gold standard)
\item
  Evaluate expected versus observed, evaluate regression line against
  empirical data
\item
  Next time: simulate new data from your regression and evaluate it
  against the observed
\end{enumerate}
\end{frame}




\end{document}
